\chapter{Теоретический обзор} \label{chapt11}

\section{Влияние нано- и микромасштабных частиц на атмосферу} \label{sect1_1}

Экологическая роль нано- и микромасштабных пылевых частиц в атмосфере в последнее время изучается все с большей широтой. Эти частицы и более крупные частицы, составленные из них, существенно влияют на климатические изменения, транспорт загрязнителей окружающей среды и биологических веществ, на оптические и электрические свойства атмосферы. 
Хотя основным источником пылевых частиц является поднимаемая ветром минеральная пыль и морская соль, извержения вулканов, обрушение крупных каменных лавин, большой вклад делают также и техногенные выбросы пыли в промышленных регионах мира при разработке полезных ископаемых, добычи энергоносителей и производстве энергии.
Условно, в зависимости от размера, атмосферную пыль разделяют на три категории [6,7]:
•	Ядра Айткена (диаметр менее 0.1 мкм);
•	Частицы среднего размера (диаметр от 0.1 мкм до 2.5 мкм);
•	Крупные частицы (диаметр более 2.5 мкм).
Кроме этого, в настоящее время выделяют сверхмелкие частицы с размером менее 10 нм.
Разделение связано с тем, что характерные свойства частиц, источники, стоки и их химический состав варьируются в зависимости от размера частиц.
Химический состав частиц может быть самым разнообразным и зависит от исходных минералов и способов создания частиц.
Важными характеристиками атмосферных пылевых частиц является их время жизни в атмосфере и влияние на неё. 
Основными видами осаждения частиц из атмосферы являются – диффузия, гравитационное осаждение и влажное осаждение. Каждый из типов осаждения характерен для частиц определенного размера. Для частиц с диаметром менее 50 нм характерен диффузионный механизм выпадения. Частицы могут оседать, диффундируя до поверхности Земли (сухое осаждение), сталкиваясь с частицами больших размеров (межфазная коагуляция) или вырастая до больших размеров за счет конденсации газов. 
Время жизни частиц в атмосфере может изменяться от нескольких минут до нескольких недель в зависимости от диаметра и площади поверхности частицы. Для крупных частиц основным механизмом оттока из атмосферы является гравитационное осаждение из-за относительно больших размеров. Самое большое время жизни у частиц средних размеров, поскольку они не удаляются эффективно ни диффузией, ни гравитационным осаждением и живут в атмосфере от нескольких дней до нескольких недель.
Действие частиц нано- и макромасштабных размеров на окружающую среду весьма разнообразно за счет влияния на оптические свойства атмосферы – начиная от ухудшения видимости из-за рассеяния и поглощения света и заканчивая серьезными климатическими изменениями. Воздействие пыли на атмосферу заметно различается в зависимости от её химического состава и размера частиц.
Мелкие частицы дают большой вклад в понижение видимости, т.к. их размер сравним с длиной волны видимого света, а вклад сверхмелких и наномасштабных частиц не велик.
Нано- и микромасштабные частицы в атмосфере могут приводить к уменьшению глобальной температуры из-за отражения солнечного излучения и влияния на оптические  свойства облаков и их время жизни. Также возможна задержка длинноволнового излучения Земли. 
В совокупности эти эффекты неоднозначно проявляются на климате Земли, действуя порой противоположно друг другу. Но они ещё недостаточно хорошо исследованы и нуждаются в дальнейшем изучении.
По современным оценкам, около десяти процентов от общей массы аэрозолей имеют антропогенную природу; при этом большой вклад имеют открытые химические карьерные взрывы [8]. Также одним из возможны источников частиц нанодиапазона в атмосфере считаются крупномасштабные обрушения скальных склонов – оползни. Объем породы, подвергающийся разрушению, достигает миллиардов кубических метров. Причем суммарный объем частиц с размерами менее микрометра составляет 3-5\% от общего объема разрушенной породы, т.е. от нескольких сотен до десятков миллионов кубических метров. По свидетельствам очевидцев, крупные оползни сопровождаются массовым выбросом пыли [9].

\section{Достижение теоретического предела прочности частиц} \label{sect1_2}

Большая роль нано- и микромасштабных частиц в атмосфере по-новому затрагивает вопрос о достижении теоретического предела прочности частиц при уменьшении их размера, так как количество и размерно-массовое соотношения частиц после измельчения определяются такими свойствами исходного материала как прочность. Фактический предел прочности — нагрузка, при которой происходит хрупкое разрушение материала — заметно отличается от теоретически рассчитанного из параметров кристаллической решетки значения, иногда разница соответствует нескольким порядкам величины. Такое поведение прочности связывают с наличием различных нарушений упорядоченной кристаллической структуры вещества —  микротрещин и дислокаций,— которые выступают в роли концентраторов напряжений, увеличивая напряжение его на своих границах в сотни раз, и, реализуя локально теоретическую критическую нагрузку. Широкий разброс прочности для одного минерала или горной породы как раз и можно объяснить индивидуальным набором дефектов для каждого образца.
Таким образом, с уменьшением размера частицы, вероятность наличия дефектов кристаллической структуры уменьшается, следовательно можно ожидать увеличения её прочности. 
Классическими стали опыты Гриффитса по вытягиванию тонких стеклянных волокон в 20-х – 30-х годах. Чем тоньше были полученные нити, тем они оказывались прочнее. Сначала их прочность увеличивалась медленно, но по мере того, как они становились очень тонкими, прочность возрастала весьма быстро. Прочность волокон диаметром около 2,5 мкм сразу после вытягивания составляла 600 кг/мм2 и более, а спустя несколько часов падала примерно до 350 кг/мм2. Кривая зависимости прочности от диаметра волокна росла столь стремительно, что трудно было установить верхний (максимальный) предел для величины прочности.
Гриффитс не мог ни изготовить, ни испытать волокна тоньше примерно 2,5 мкм. Однако он экстраполировал кривую "прочность-размер" в область ничтожно малых толщин, и оказалось, что прочность тончайших нитей должна быть около 1100 кг/мм2. Вычисленная величина прочности для его стекла была чуть меньше 1400 кг/мм2. Поэтому Гриффитс сделал вывод, что ему практически удалось приблизиться к теоретической прочности, и, если бы на самом деле можно было сделать более тонкие волокна, их прочность была бы очень близка к теоретической.
Также известны опыты А. Ф. Иоффе по разрыву кристаллов каменной соли, в которых показано влияние состояния поверхности образца (наличие трещинок, царапин), а также среда, в которой он находится, на прочность. А. Ф. Иоффе измерял прочность кристаллов каменной соли на воздухе и при погружении в воду, и оказалось, что она увеличивается с 0,5 до 160 кг/мм2. Такое изменение можно объяснить растворением в воде приповерхностного слоя кристаллов и ликвидацией дефектов этого слоя.
В целом, как отмечал Гриффитс, дефекты на поверхности образца играют более важную роль в разрушении образца, чем внутренние.
Ряд других работ также затрагивает вопрос достижения теоретической прочности. Например, в работе М. И. Койфмана [10] исследуется прочность кварца, корунда, искусственного корунда и карбида кремния в зависимости от размеров зерен, и устанавливается постепенное увеличение прочности при уменьшении размеров зерен. Минимальный размер зерен в опытах составил 90 мкм, что для карбида кремния дало увеличение прочности в 10 раз.
Современные упоминания предела прочности связаны с получением идеальных наномасштабных структур из атомов кремния со свойствами, которые нельзя назвать ни хрупкими, ни пластичными [11]. Такие образцы, так как есть ещё дополнительная возможность определять их форму, обнаруживают очень высокие прочностные характеристики.
Опытные данные о зависимости прочности образцов от их размеров привели к созданию статистической теории прочности, которая находит себе применение в целом ряде вопросов [12].
Основные тезисы этой теории в том, что во всяком образце данного материала имеется некоторое количество дефектов, снижающих его прочность, причем предполагается, что дефекты распределены равномерным способом и известна их объемная концентрация   в единице объема  . Далее определяется степень опасности разных дефектов при разных напряжениях. 
Таким образом, вероятность того, что фактическая прочность образца лежит в единице интервала прочностей вблизи данного значения z, может быть вычислена с помощью формулы:
 ,
где  — величины постоянные для данного материала.
Наиболее вероятное значение этой прочности определяется условием
 .
Также было обнаружено, что прочность зависит не только от объема, но и от времени в течение которого она измерялась[13]. А именно, прочность одного и того же образца оказывается тем меньше, чем дольше он подвергается растяжению. Так, например, в случае тонких стеклянных нитей при одних и тех же условиях прочность увеличивается в три раза при сокращении времени нагрузки от 24 часов до 0,01 сек. Это обстоятельство показывает, что дефекты, обуславливающие уменьшение прочности, не остаются неизменными во времени, но что при приложении нагрузки они становятся все более и более опасными.
Закономерности распределения частиц пыли и порошкообразного материала
Промышленные пыли и продукты измельчения материалов состоят из частиц, имеющих в подавляющем большинстве случаев неправильную геометрическую форму, и обычно являются полидисперсными системами.
Для оценки степени дисперсности таких материалов могут быть использованы различные характеристики, например, наименьший и наибольший размер частиц, разность между наибольшим и наименьшим размерами, средний размер частиц, удельная поверхность и др.
Однако наиболее полно дисперсность характеризуется дисперсным (гранулометрическим, зерновым) составом. При такой характеристике устанавливаются не только перечисленные выше параметры, но и процентное содержание частиц каждого размера. 
Они могут быть одно- или многокомпонентными. Свойства пылевидного материала удобно описывать функцией распределения  массы материала по диаметрам частиц или связанной с ней функцией  .
Функция  равна отношению массы всех частиц, диаметр которых меньше  , к общей массе пылевидного материала. Функция   определяется как отношение массы всех частиц, диаметр которых больше  , к общей массе материала.
Истинное зерновое распределение измельченного материала зависит только от условий его образования. Однако получаемые в результате эксперимента распределения одного и того же материала различны в зависимости от применяемого метода дисперсионного анализа. При этом, как правило, кривая распределения и кривая плотности распределения, построенные по результатам анализа, выполненного одним методом, отличаются от кривых, построенных по результатам, полученным другими методами. Это объясняется тем, что каждый метод обладает своими систематическими ошибками, которые вызваны допущениями, лежащими в основе метода.
Для аналитического описания кривых распределения и плотности распределения измельченных материалов были предложены различные формулы. Формула каждой конкретной кривой получается путем подстановки в предлагаемые уравнения значений нескольких параметров, устанавливаемых экспериментальным путем — по результатам анализа дисперсного состава. 
Различные формулы, выражающие функции распределения, приведены в ряде монографий [14,15,16]. Они могут быть одно-, двух- и трех параметрическими, и подразделены на теоретические и экспериментальные зависимости. 
Теоретические формулы выведены на основе некоторых физических представлений о закономерностях распределения частиц пыли и порошкообразных материалов. Эмпирические получены на основе описания результатов дисперсионных анализов. 

\section{Теоретические формулы } \label{sect1_3}

Логарифмически нормальное распределение 
Логарифмически нормальное распределение (ЛНР) получается, если в нормальную Гауссову функцию распределения подставить в качестве аргумента не диаметр частиц, а логарифм диаметра. 
Кривая такого распределения имеет Гауссову форму, если по оси абсцисс откладывать логарифмы диаметров частиц, а по оси ординат — значения D или R. В этом случае значения диаметров, проставляемые через равные интервалы оси абсцисс, возрастают в геометрической прогрессии (так как прибавление логарифма соответствует умножению подлогарифмического выражения на постоянное число). Кривая плотности распределения при этом имеет симметричный вид. 
Справедливость логарифмически нормального закона для всех случаев, когда мы имеем дело с частицами вещества, полученными механическим измельчением в течение длительного времени, теоретически доказана Колмогоровым. Применимость этого закона для многих видов пыли и порошкообразных материалов подтверждается рядом экспериментальных исследований. 
При теоретическом исследовании помимо достаточно большого времени измельчения   были сделаны следующие основные допущения: 
1) за промежуток времени между   и   из одной частицы диаметром   получается при дроблении несколько частиц, распределение которых по отношению   не зависит от абсолютных размеров первоначальной частицы, от ее предшествующей истории (т. е. от того, в результате каких предшествующих дроблений она возникла) и от судьбы других частиц; 
2) среднее число частиц, получающихся из одной частицы за промежуток времени между   и   конечно и больше единицы. 
Функция ЛНР массы материала по диаметрам частиц имеет вид: 
 ,
где   — медиана распределения;   — стандартное (среднеквадратическое) отклонение логарифмов диаметров от их среднего значения [42]. 
Интеграл, входящий в уравнение не может быть выражен через элементарные функции. Для вычисления искомой функции ее преобразуют в функцию нового аргумента t: 
 
Аргумент t называется нормированной нормально распределенной величиной. Среднее значение этой величины равно нулю. 
Произведя указанную замену, получают функцию аргумента t:
 ,
 которая называется нормированной функцией нормального распределения и изменяется в пределах от 0 до 1.
Логарифмически нормальное распределение удобно изображать графически на логарифмически вероятностной координатной сетке, т. е. в такой прямоугольной системе координат, по оси абсцисс которой откладываются логарифмы диаметров, а по оси ординат откладываются значения величины t. 
Вычерченный на такой сетке график ЛНР изобразится прямой линией, поскольку будет выражать зависимость t от  , которая, как это следует из выражения является линейной. 
Угловой коэффициент  этой прямой равен  ; здесь   — угол между прямой и положительным направлением оси абсцисс. 
Чем более полидисперсен порошкообразный материал, тем больше дисперсия и соответственно стандартное отклонение   ее кривой распределения, а, следовательно, тем меньше угол  . Чем ближе к вертикали линия распределения в логарифмически вероятностной координатной сетке, тем уже распределение, т. е. тем более однороден по своим размерам исследуемый материал. 
Более измельченному материалу соответствует меньший медианный диаметр   и более высокое расположение линии распределения на логарифмически вероятностной сетке. 
Формулы Загустина 
Фаренволд и Загустин принимали, что весьма распространенный закон Максвелла, говорящий, что при многих физических процессах изменение во времени интересующего нас параметра пропорционально самому параметру, применим к процессу дробления материалов в следующем виде: скорость уменьшения массы каждой фракции прямо пропорциональна массе этой фракции в мельнице [17], т. е. 
 ,
где m — масса фракции крупнее  . 
Загустин принял, что  , а также, что при дроблении отдельного куска одним ударом получается продукт, имеющий функцию распределения 
 
где  — диаметр исходного куска, k — постоянная, характеризующая рассыпаемость материала при разрушении (дроблении). 
На основе этих предположений он вывел сложное общее уравнение характеристики измельченного материала в зависимости от времени, которое для частных случаев принимает следующие выражения. 
Для k = 0 (кварц и некоторые другие минералы, для которых k весьма мало) распределение массы по диаметрам после измельчения в течение времени   описывается функцией 
 ,
где  — функция, характеризующая распределение исходного материала; a —постоянная, входящая в исходное равенство  .
Для   и больших значений   
 
Для   и малых значений   ( < 1) функция распределения D = 1 — R равна 
 
Эмпирические формулы 
Пригодность эмпирических формул устанавливается путем сопоставления получаемых по ним расчетных данных с результатами опыта или путем сравнения расчетных и экспериментальных кривых 
Ввиду того, что исследователи изучали измельчение разнообразных материалов в различного типа помольных и дробильных агрегатах и зерновое распределение устанавливали по результатам анализов дисперсного состава, выполненных с применением разных приборов, предложенные формулы не являются общими, охватывающими все виды измельчения и анализа порошкообразного материала. Каждая из них описывает статистические распределения, отвечающие конкретным условиям измельчения и методу определения фракционного состава. 
Формулы Мартина—Андреасена 
Впервые продукт измельчения рассматривается как статистическая совокупность частиц различных размеров в работах Мартина. Он исследовал измельченный кварцевый песок. Размеры частиц определялись методом микроскопии. В первой серии опытов был взят диапазон 1,4—20 мкм, во второй — 10—50 мкм. Мартин предложил аппроксимировать экспериментально полученные кривые плотности распределения числа частиц по диаметрам следующей формулой (в принятых обозначениях): 
 ,
где  — частота наблюдения частиц, диаметры которых лежат в интервале  , отнесенная к величине этого интервала; n — полное число частиц в просмотренном материале; b — постоянная, определяемая из опыта. 
Удобство этой формулы заключается в том, что в полулогарифмических координатах, когда на оси абсцисс откладываются размеры частиц  , а на оси ординат — логарифмы значений  плотность распределения изобразится прямой, определяемой уравнением 
 
Андреасен, приняв во внимание, что масса частицы пропорциональна кубу ее диаметра (характерного линейного размера), предложил распространить формулу Мартина на 
распределения массы продуктов помола. 
Формула Розина—Раммлера 
Розин и Раммлер, рассматривая зерновое распределение продуктов измельчения как статистическую совокупность, нашли, что кривые распределения по данным ситовых анализов могут быть выражены уравнением 
 
где   и n — постоянные, легко определяемые в логарифмической форме этого уравнения по опытным данным. 
По своему физическому смыслу   представляет собой такой диаметр, при котором масса частиц крепнее   составляет 36,8%, а мельче   — 63,2%.
Показатель степени n характеризует ширину распределения, т.е. степень однородности материала по размерам частиц: чем больше n, тем уже диапазон размеров частиц (при   все частицы имеют размер  ).
Формула Розина—Раммлера подобрана на основании кривых Пирсона, оказавшихся наиболее подходящими для выражения функциональных зависимостей, установленных из опыта. 
В одной из более поздних работ, посвященных распределению измельченных продуктов [18], Раммлер указывает, что приведенные выше формулы Розина—Раммлера не являются универсальными, а лишь приближенными, но применимы при многих способах измельчения. Он приводит несколько примеров, показывающих, что при некоторых видах измельчения плотность распределения значительно точнее описывается гауссовской кривой (металлический порошок, полученный путем распыления жидкого металла; порошки, полученные помолом при сушке распылением, при конденсации из пара и др.). 
Для практического применения формулы Розина—Раммлера показательное уравнение дважды логарифмируется: 
 
Это уравнение описывает прямую в координатах  ,  , т. е. в двойной логарифмической координатной сетке.
Распределение порошкообразных материалов, подчиняющихся закономерности Розина—Раммлера, должно в этих координатах изображаться прямыми. Параметр n, характеризующий ширину распределения (дисперсию), приобретает, согласно уравнению, значение тангенса угла наклона прямой. 
Краткий анализ и рекомендации по применению формул для описания зернового распределения 
Во всех случаях, где это возможно, желательно эмпирические данные аппроксимировать одно- и двухпараметрическими формулами указанными выше. К тому же в большинстве случаев они могут быть представлены в виде прямых, параметры которых очень просто определяются графически.
Одно- и двухпараметрическими формулами аппроксимируются эмпирические кривые распределения, полученные в результате анализов дисперсного состава некоторых продуктов измельчения, выполненных определенными методами. Из них практическое распространение получили формулы Розина- Раммлера. Остальные формулы этой группы в практике дисперсионных анализов почти не применяются. 
Формула Розина—Раммлера получила весьма широкое распространение, хотя Раммлер, как отмечалось выше, считает ее не универсальной, а лишь приближенной, применимой при некоторых способах измельчения. 
Аналитические выражения кривых распределении могут быть в ряде случаев использованы для экстраполяции распределения по крупности частиц за пределы области, определяемой анализом. Однако такая экстраполяция не может считаться достаточно достоверной, так как вне пределов эксперимента возможны отклонения от найденных аналитических зависимостей. Основной причиной таких отклонений является то, что измельчение происходит со значительными отклонениями от теоретической схемы. Вследствие этого нельзя быть всегда уверенным в том, что закономерность распределения, экспериментально установленная для некоторого промежутка значений размеров частиц, сохраняется и вне этого промежутка. Отклонения вызываются главным образом тем, что измельчение (помол) складывается из двух процессов — истирания и  раскалывания, причем раскалывание происходит по плоскостям, разделение по которым требует меньших усилий. Такие слабые места структуры имеются в громадном большинстве материалов, подвергаемых измельчению. Даже в стеклах и хорошо образованных кристаллах, например в кварце, имеются многочисленные дефекты структуры — пустоты, включения, внутренние и поверхностные трещины. При продолжающемся измельчении приходится прикладывать все большую силу. Однако наступает момент, когда увеличение давления не приводит к дальнейшему измельчению. Поэтому при любом размоле возникает нижний предел размера зерен, что аналитическими формулами не учитывается. Эта причина отклонений от аналитических зависимостей является общей как для эмпирических формул, так и для формул, основанных на некоторых физических представлениях о процессе измельчения. 
Второй причиной, могущей вызвать отклонения экспериментально найденных кривых распределения от ожидаемых согласно аналитическим формулам, являются систематические ошибки, присущие примененному методу анализа дисперсного состава. Как очевидно, желательно, чтобы аналитические формулы для кривых распределения вытекали из определенных физических представлений о природе образования полидисперсного материала, достаточно логичных и вероятных. Тогда отклонение данных анализа дисперсного состава от аналитических выражений можно объяснить либо тем, что процесс образования материала не соответствует схематическому представлению о нем и поэтому истинное распределение плохо описывается принятой формулой, либо систематическими ошибками метода анализа, либо, наконец, совместным влиянием обеих причин. Какая из названных причин повлияла на расхождение между опытной и расчетной кривыми распределения, можно выяснить, проводя анализ дисперсного состава несколькими методами.
На определенных представлениях о механизме образования полидисперсных материалов основано, как было указано выше, логарифмически нормальное распределение. 
Вероятным представляется и допущение Фаренволда и Загустина о применимости к процессу дробления закона Максвелла. Однако применение полученных Загустиным функциональных зависимостей затруднительно, вследствие чего они не получили распространения в практике. 
Схема процесса дробления, принятая Колмогоровым при математическом обосновании применимости логарифмически вероятностного закона к зерновому распределению продуктов помола, представляется логичной для тонкоизмельченных материалов и пыли. 
Следует отметить вслед за Самсоновым [19], что ЛНР приемлемо для описания ненарушенного зернового состава раздробленных материалов, если же состав нарушен просеиванием или аэродинамическими процессами, связанными с витанием пыли в воздухе, то нет уверенности в его применимости. 
Приведенный анализ закономерностей зернового распределения однокомпонентных полидисперсных материалов позволяет сделать следующие практические выводы. 
1. Наиболее обоснованным для тонкоизмельченных однокомпонентных материалов и пыли является ЛНР. Однако распределения, получаемые в результате анализа дисперсного состава порошкообразных материалов и пыли, могут и не подчиняться логарифмически нормальному закону. Отклонения могут вызываться как механизмом образования полидисперсного материала, частности нарушением его зернового состава уже после окончания процесса измельчения, так и систематическими ошибками метода дисперсионного анализа, вызывающими отклонение наблюдаемого распределения от истинного. 
2. Во всех случаях, где это окажется возможным, целесообразно зерновое распределение однокомпонентных полидисперсных материалов, определяемое по результатам анализа дисперсного состава, выражать: 
а) для продуктов измельчения — с помощью ЛНР, формул Розина—Раммлера; 
б) для пыли, витающей в относительно спокойном воздухе или в другой среде, — с помощью формулы Ромашова[16]. 
Чтобы выяснить, какая из этих формул наиболее точно описывает распределение, полученное при анализе дисперсного состава, экспериментальные точки следует наносить на логарифмически вероятностную сетку, на двойную логарифмическую сетку и т.д. Спрямление кривой распределения в одной из названных систем координат будет свидетельствовать о применимости соответствующих формул. 
В заключение необходимо отметить работы Петролля [20], который на основе исследования закономерностей процесса измельчения материалов предложил теоретически обоснованную четырехпараметрическую функцию и показал, что широко применяемые формулы Розина—Раммлера, ЛНР и др. являются ее частными случаями.


\clearpage