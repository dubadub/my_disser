\chapter*{Заключение}						% Заголовок
\addcontentsline{toc}{chapter}{Заключение}	% Добавляем его в оглавление

Основные результаты работы заключаются в следующем.
\begin{enumerate}
  \item Разработана методика отбора и анализа проб нано- и микромасштабных частиц в геосистемах, нашедшая применение при экспериментальном воспроизведении оползневого дробления,  натурных экспериментах в гранитном массиве Фенноскандии около г. Выборга и на Лебединском ГОКе, исследовании частиц, составляющих Атмосферные Коричневые Облака и т.д. Методика отбора и анализа проб нано- и микромасштабных частиц в геосистемах включает следующие положения: для сбора нано- и микромасштабных частиц  используются пластмассовые контейнеры размером $80\times50$~мм\textsuperscript{2} и высотой 40 мм, в которых содержатся фильтры Петрянова АФА-РСП-10 площадью примерно 5~см\textsuperscript{2} каждый, а также двухсторонний скотч. Предложено размещать разработанные контейнеры для сбора нано- и микромасштабных частиц на достаточной высоте для минимизации эффекта попадания посторонней пыли (например, пыли, поднимаемой с поверхности грунтовых дорог автотранспортом). Для определения фонового выпадения пыли проводятся аналогичные замеры пыли при отсутствии пыли от исследуемых источников. Анализ проб нано- и микромасштабных частиц в геосистемах подразумевает использование стандартных методов исследования распределений частиц по размерам и их химического состава с помощью электронных и оптического микроскопов, дифракции рентгеновских лучей, а также рентгеновской флюоресценции. Разработано программное обеспечение для получения данных о наномасштабных частицах в геосистемах (в том числе распределений по размерам), изображения которых получены с помощью электронного микроскопа. При работе с ним определяется масштаб и площадь снимка. Затем выделяются частицы, размером входящие в заданный заранее диапазон размеров. В дальнейшем проводится обобщение результата обработки снимка путем пересчета индивидуальных размеров в данные по количеству частиц на заданные заранее интервалы.
  \item Осуществлено экспериментальное воспроизведение оползневого дробления с сохранением исходной макроструктуры оползневых тел. Показано, что многократное последовательное дробление горной массы при сочетании умеренных литостатических нагрузок и сдвиговых деформаций, воспроизведенное в экспериментах, позволяет объяснить наблюдаемое в природе сильное дробление нижних частей оползневого тела без перемешивания материала, а также более высокую эффективность «оползневого дробления» по сравнению с дроблением взрывом и механическим дроблением. Воспроизведение «оползневого дробления» в промышленных масштабах может позволить сократить расход электроэнергии и может оказаться экономически эффективным. 
  \item Проведено исследование образования нано- и микромасштабных частиц при разрушении скальной породы. Для этой цели осуществлены лабораторные эксперименты с образцами гранита, а также натурные эксперименты в гранитном массиве Фенноскандии около г. Выборга. В лабораторных экспериментах исследовалось образование мелкодисперсных частиц при разрушении образцов на излом, на сдвиг и на кручение. Показано, что ... раскрыть. Натурные экперименты имели целью исследование мелкодисперсных частиц нано- и микромасштабного размера, образующихся в результате разрушения горных пород как однократными, так и многократными взрывами химических ВВ. Выполнен статистический анализ мелкодисперсной фракции по размерам, и проведено сравнение с классическими распределениями Колмогорова и Розина–Раммлера. Доказана возможность образования при разрушении скальной породы наномасштабных частиц с размерами, превосходящими 20 нм. Показано, что с ростом числа взрывов роль многократного дробления возрастает. Значительная часть наномасштабных частиц формируется вследствие дробления более крупных частиц, образованных при первом взрыве, во время последующих взрывов. Что-то написать о свойствах, сходстве и различии частиц в лабораторных и натурных экспериментах. В пределах точности экспериментов не выявлено усиления прочности материала при его дроблении до наномасштабных размеров.
  \item Проведены натурные эксперименты, имеющие целью исследование мелкодисперсных частиц нано- и микромасштабного размера, образующихся в результате разрушения горных пород (каких - в смысле, что они отличаются от скальных пород, описанных в предыдущем пункте) в результате массового взрыва на Лебединском ГОКе. Выполнен статистический анализ мелкодисперсной фракции по размерам, и проведено сравнение с классическими распределениями Колмогорова и Розина–Раммлера. Получены данные о минералогическом и гранулометрическом составе пылевого облака, возникшего после массового взрыва на Лебединском ГОКе. Показано, что основная часть пыли состоит из частиц разнообразной морфологии железистого кварцита. Обнаружены наномасштабные частицы, образованные в результате разрушения породы взрывом, с размерами, превосходящими 60 нм. Гранулометрический состав пыли характеризуется малым количеством крупных частиц. Распределение по размерам для частиц, меньших 2.5 мкм, хорошо соотносится с распределением Розина-Раммлера. 
\end{enumerate}


\clearpage
