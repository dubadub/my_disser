\chapter{Теоретический обзор} \label{chapt1}

\section{Получение распределения частиц по размерам} \label{sect1_1}


Размерно-массовое распределение частиц в диапазоне от десятков нанометров до миллиметров — важная характеристика грунта, с помощью которой осуществляется анализ его некоторых свойств и особенностей происхождения.
Получение распределения в данной работе разбито на два этапа. 
Сначала образец, для которого строится массово-размерное соотношение, последовательно просеивается через набор стандартных сит с размерами ячей 2,5; 1,6; 1; 0,63; 0,4; 0,315; 0,2; 0,16 и 0,06 мм. Все полученные фракции взвешиваются и затем строятся предварительные распределения массового количества частиц по размерам для диапазона 0,06—2,5 мм в координатах Розина–Раммлера.
После отсева, частицы пылевидной фракции с размером менее 60 мкм анализируются на растровом электронном микроскопе. По анализу изображений частиц в разных масштабах строится распределение по размерам в области от 50 нм до 60 мкм и затем эти данные сшиваются с данными ситового рассева.
Получение распределения частиц по размерам с помощью электронного микроскопа
Общие технические характеристики микроскопа JEOL 6460 LV, с помощью которого были получены изображения частиц, представлены в таблице.
Разрешение	3,0 нм (З0 кВ, рабочее расстояние 8 мм, изображение во вторичных электронах в режиме с высоким вакуумом)
Увеличение	От 5 до 300 000 (136 шагов)
Ускоряющее напряжение	От 0,3 до 30 кВ (55 шагов)
Реальная разрешающая способность прибора, которой удалось достичь, ниже заявленной в описании из-за особенностей исследуемого материала. Идеальным объектом для наблюдения в электронный микроскоп является проводящая пыль на проводящей подложке с максимально большой разницей атомного номера вещества пыли и вещества подложки. Для анализа непроводящего образца обычно наносят на него тонкий слой проводящего вещества (обычно напыляют золото или углерод). Из-за диэлектрических свойств пыли, при бомбардировке электронным пучком её частицы заряжаются, и возникают помехи из-за влияния заряда на траекторию вторичных электронов, а так же начинается движение самих частиц пыли. Для анализа непроводящих веществ на данном микроскопе присутствует режим низкого вакуума (LV), в котором обеспечивается отвод заряда от образца воздухом, но при этом максимальное разрешение заметно падает. В данном режиме удалось наблюдать частицы с минимальным размером порядка 1 мкм. Поэтому пришлось выбрать для исследований режим высокого вакуума с регистрацией вторичных электронов, но с низким ускоряющим напряжением (порядка 3 kV, наиболее оптимально использование 30 kV), что тоже снизило разрешение, но всё-таки удалось наблюдать частицы размером до 60 нм.
Методика приготовления образца для анализа включает в себя нанесение пыли на двусторонний проводящий скотч приблизительно в один слой. Отмечается, что прямой контакт со скотчем имеют в основном крупные частицы — от 2 мкм, а мелкие обнаруживаются только на крупных. Также, часто, во время наведения, при сильном увеличении, частица заряжается и начинает движение, иногда довольно резко, так как крепление между частицами обычно слабое.
Общая схема работы на микроскопе состоит из следующих этапов: 
1.	подготовка микроскопа — включение, откачка, прогрев катода; 
2.	подготовка образца — нанесение на специальный столик двустороннего проводящего скотча с пылью; 
3.	настройка микроскопа в начале работы — ток катода, параметры электронной линзы, положение диафрагмы, 
4.	получение изображения образца — последовательное увеличение с настройкой изображения, которая заключается в постоянном выполнении следующих операций до получения необходимого качества:
•	подстройка фокусного расстояния 
•	подстройка стигматизма для исправления формы поперечного сечения пучка электронов
•	размагничивание линзы
Если необходимо, то изменяется напряжение ускорения электронов или диаметр пучка электронов. При этом общая закономерность такова — чем меньше размер пучка, тем лучше разрешение. Но в режиме просмотра тогда много шумов и возникают сложности с установкой фокусного расстояния и выправления стигматизма, так как теперь приходится после каждой настройки смотреть на результат изменения в режиме с большим накоплением (когда происходит усреднение шума) в котором изображение обновляется за несколько десятков секунд. Когда получена хорошая настройка, выполняется максимально качественное сканирование, и изображение сохраняется на жесткий диск.
5.	после завершения съёмки — выключение микроскопа.
В итоге, после дополнительной графической оптимизации изображений, на каждый образец имеется 40-60 снимков. Обычно это несколько обзорных планов (длина масштабного отрезка — 20 – 100 мкм), 5-10 планов средней приближенности (1 – 20 мкм) и остальное — крупные (0,06 – 1 мкм). Такое распределение необходимо для того, чтобы при подсчете количества частиц, число частиц разных масштабов было примерно одинаково.
Обработка изображений для получения сведений о количестве и размерах частиц
Для удобства получения информации о размерах частиц была специально реализовано программное обеспечение.
Основная задача программы – сбор информации о количестве и размерах частиц с разных снимков и получение итогового распределения по размерам. Итоговое распределение получается благодаря сведению всех снимков в один масштаб, путем умножения количества частиц на отношение единичной площади к площади снимка.
Масштаб изображения определяется из информационных файлов, сопровождающих файл изображения. Перед началом работы с конкретным изображением, определяются интервалы размеров частиц, которые будут фиксироваться. Это необходимо для того, чтобы фиксировать только те частицы, для которых разрешения достаточно. Затем пользователь, рисуя мышью прямоугольники, выделяет частицы, и программа фиксирует их размер, который вычисляется из размера диагонали прямоугольника, разделенного на корень из 2 (таким образом, образуя альтернативный средний квадрат).
В итоге, для каждого снимка определяется распределение частиц по интервалам размеров.
Далее:
1.	выполняется сведение к эталонной площади (какое количество частиц было бы для данного диапазона на эталонной площадке)
2.	для данных одного диапазона, но с разных снимков, производится усреднение
3.	из усредненных данных сформировывается единое распределение от малых до крупных размеров
4.	осуществляется перевод количества и размеров частиц в массу всех частиц данного диапазона
5.	масса частиц всего диапазона сводится к массе остатка менее 0,06 мм, полученного после ситового анализа, путем одновременного умножения всех масс диапазонов на рассчитанный коэффициент 



\clearpage